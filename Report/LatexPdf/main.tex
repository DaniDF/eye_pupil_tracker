\documentclass[11pt]{article}

\usepackage{report}

\usepackage[utf8]{inputenc} % allow utf-8 input
\usepackage[T1]{fontenc}    % use 8-bit T1 fonts
\usepackage[colorlinks=true, linkcolor=black, citecolor=blue, urlcolor=blue]{hyperref}       % hyperlinks
\usepackage{url}            % simple URL typesetting
\usepackage{booktabs}       % professional-quality tables
\usepackage{amsfonts}       % blackboard math symbols
\usepackage{nicefrac}       % compact symbols for 1/2, etc.
\usepackage{microtype}      % microtypography
\usepackage{lipsum}		% Can be removed after putting your text content
\usepackage{graphicx}
\usepackage{natbib}
\usepackage{doi}
\usepackage[italian]{babel} %lingua principale
\setcitestyle{aysep={,}}
\linespread{1.4} %interlinea
\renewcommand{\familydefault}{\sfdefault}


\title{EYE TRACKER}

\author{Luigi di Nuzzo,
Daniele Foschi,
Filippo Veronesi\\
\AND
\AND
\AND
\AND
\AND
	Dipartimento di Informatica - Scienza e Ingegneria (DISI)\\
\AND
Alma Mater Studiorum - Università di Bologna\\
}

% Uncomment to remove the date
\date{Aprile 2022}

% Uncomment to override  the `A preprint' in the header
\renewcommand{\headeright}{Progetto e Attività Progettuale Sistemi Digitali M}
\renewcommand{\undertitle}{Progetto e Attività Progettuale Sistemi Digitali M}
\renewcommand{\shorttitle}{}

%%% Add PDF metadata to help others organize their library
%%% Once the PDF is generated, you can check the metadata with
%%% $ pdfinfo template.pdf
% \hypersetup{
% pdftitle={A template for the arxiv style},
% pdfsubject={q-bio.NC, q-bio.QM},
% pdfauthor={David S.~Hippocampus, Elias D.~Striatum},
% pdfkeywords={First keyword, Second keyword, More},
% }

\begin{document}
\maketitle

\newpage
\tableofcontents
\thispagestyle{empty}

\newpage






\newpage
\setcounter{page}{1}
\section{Introduzione}
Questo progetto prevede la realizzazione di un applicativo Android che, sfruttando una rete neurale,
sia in grado di riconoscere il volto dell’utente e di questo ultimo, anche i propri occhi. Il sistema sfrutterà un modello di rete neurale
addestrato in modo tale da essere in grado di riconoscere i visi e gli occhi di uno o più utenti grazie ad un dataset di 10 mila immagini di visi umani.
L’applicativo permetterà sia di svolgere un filtraggio "live", cioè aprendo la camera frontale ed esterna dello smartphone
e individuando real-time la faccia e gli occhi dell’utente, !!!! sia uno “statico” in cui si selezionano dalla galleria
una foto e in output vengono riconosciuti gli occhi !!!! DA VEDERE SE METTERE. !!!!!
\newline
Inoltre, è stato implementato un gioco (SPIEGAZIONE) aiutato da un tool di calibrazione.
\newline \newline
Tale caso di studio è un classico esempio di applicazione di machine learning e il software farà ricorso
a una rete neurale convoluzionale (CNN). Tale scelta è dovuta al fatto che una rete neurale
rappresenta il modo più comodo e pratico per problemi di object detection, come quello di questa
attività in cui vengono individuati gli occhi.
\newline \newline
L’applicativo, inoltre, è pensato per la piattaforma Android e quindi tale progetto pone attenzione
anche all’uso di risorse in quanto dovrà funzionare su smartphone, ovvero dispositivi embedded.

\newpage

\section{Rete Neurale}
L’obbiettivo primario di questo task è produrre un modello di rete neurale addestrata in grado di
riconoscere gli occhi di una o più persone. A tal scopo si è utilizzato un modello messo a disposizione dalla libreria
TensorFlow ottimizzato per il training di modelli specifici per object detection di immagini. Visto
l’ambito dei sistemi embedded nel quale il progetto complessivo si colloca si è optato per ResNet e MobileNet,
una CNN pensata per dispositivi mobile. MobileNet è il primo modello di computer vision pensato
per dispositivi embedded basato su TensorFlow. MobileNet è sufficientemente leggera e veloce da
essere eseguita su smartphone senza consumo di risorse eccessivo mantenendo comunque una
precisione adeguata.


\subsection{Dataset}
Per il training del modello si è utlizzato un dataset pubblico rilasciato da Google (LINK DA INSERIRE) contenente 10 mila immagini di visi umani, dove ogni foto conteneva da uno a più visi umani.
\newline
Oltre alle immagini il dataset conteneva un utile file Excel che indicava le coordinate della posizione degli occhi delle persone all'interno dell'immagine.
\newline \newline
Prima dell'addestramento della rete era però necessario convertire questi dati di input in file XML in modo tale da poterli poi convertire in TFRecord, utili per Tensorflow. Per questo motivo è stato implementato uno script Java in grado di creare un file XML per ogni foto del dataset con al suo interno le informazioni geografiche della posizione degli occhi. 


\subsection{Training}
Si è quindi proceduto all’addestramento della rete
tramite Python usando TensorFlow e le API di Keras. Si è ottenuto in output un modello addestrato
e pronto all’uso. 
\newline \newline
Prima di testare direttamente su Andorid, abbiamo deciso prima di testare e analizzare la nostra nuova rete utilizzando la libreria Matplotlib, utile per visualizzare graficamente via shell i nostri risultati. In particolare abbiamo usato Tkinter, l'unico framework GUI incluso nella libreria standard di Python.
A fini di testing si è preso ad esempio il volto di un
personaggio pubblico, quello di Donald Trump (INSERIRE FOTO GUI tkAgg PUTIN O ALTRI MEGLIO).


\subsection{TfLite}
Dopo esserci accertati che la rete funzionasse correttamente e avesse dei livelli di precisione sopra una certa soglia, si è ottenuto in output un modello addestrato
e pronto all’uso, che poi è stato convertito in un formato adatto ai sistemi embedded, quello di TensorFlow Lite.


\newpage
\section{Progetto Android}
\label{sec:others}

\subsection{Nerd Mode}
Citations use \verb+natbib+. The documentation may be found at
\begin{center}
	\url{http://mirrors.ctan.org/macros/latex/contrib/natbib/natnotes.pdf}
\end{center}

Here is an example usage of the two main commands (\verb+citet+ and \verb+citep+): Some people thought a thing \citep{kour2014real, hadash2018estimate} but other people thought something else \citep{kour2014fast}. Many people have speculated that if we knew exactly why \citet{kour2014fast} thought this\dots

\subsection{Calibration}
\lipsum[10]
See Figure \ref{fig:fig1}. Here is how you add footnotes. \footnote{Sample of the first footnote.}
\lipsum[11]

\begin{figure}
	\centering
	\fbox{\rule[-.5cm]{4cm}{4cm} \rule[-.5cm]{4cm}{0cm}}
	\caption{Sample figure caption.}
	\label{fig:fig1}
\end{figure}

\subsection{Game}
See awesome Table~\ref{tab:table}.

The documentation for \verb+booktabs+ (`Publication quality tables in LaTeX') is available from:
\begin{center}
	\url{https://www.ctan.org/pkg/booktabs}
\end{center}







\newpage
\section{Conclusioni}
Gli obiettivi del progetto sono stati raggiunti a pieno e con risultati soddisfacenti. 
\newline \newline
L’applicativo è
conforme alle aspettative e svolge i compiti adeguatamente.
Le parti realizzate
in Android risultano con prestazione adeguate.
\newline \newline
Il modello addestrato restituisce un output corretto la maggior parte delle volte con una precisione di eye tracking del 95\%.

\newpage
\bibliographystyle{unsrtnat}
\bibliography{references}  %%% Uncomment this line and comment out the ``thebibliography'' section below to use the external .bib file (using bibtex) .


%%% Uncomment this section and comment out the \bibliography{references} line above to use inline references.
% \begin{thebibliography}{1}

% 	\bibitem{kour2014real}
% 	George Kour and Raid Saabne.
% 	\newblock Real-time segmentation of on-line handwritten arabic script.
% 	\newblock In {\em Frontiers in Handwriting Recognition (ICFHR), 2014 14th
% 			International Conference on}, pages 417--422. IEEE, 2014.

% 	\bibitem{kour2014fast}
% 	George Kour and Raid Saabne.
% 	\newblock Fast classification of handwritten on-line arabic characters.
% 	\newblock In {\em Soft Computing and Pattern Recognition (SoCPaR), 2014 6th
% 			International Conference of}, pages 312--318. IEEE, 2014.

% 	\bibitem{hadash2018estimate}
% 	Guy Hadash, Einat Kermany, Boaz Carmeli, Ofer Lavi, George Kour, and Alon
% 	Jacovi.
% 	\newblock Estimate and replace: A novel approach to integrating deep neural
% 	networks with existing applications.
% 	\newblock {\em arXiv preprint arXiv:1804.09028}, 2018.

% \end{thebibliography}


\end{document}
