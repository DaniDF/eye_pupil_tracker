L’obiettivo primario di questo task è quello di produrre un modello di rete neurale addestrata in grado di riconoscere gli occhi di una o più persone. Successivamente, si è prodotto un modello di rete neurale addestrata in grado di riconoscere le pupille degli esseri umani.

A tali scopi si è utilizzato un modello messo a disposizione dalla libreria \textit{TensorFlow 2 Detection Model Zoo \cite{tensorflow_repo}} ottimizzato per il training di modelli specifici per object detection di immagini. Visto l’ambito dei sistemi embedded nel quale il progetto complessivo si colloca si è optato per un modello SSD MobileNet\footnote{SSD MobileNet V2 FPNLite 640x640}\cite{mobilenet_repo}, una CNN pensata per dispositivi mobile. MobileNet è il primo modello di computer vision pensato per dispositivi embedded basato su TensorFlow. MobileNet è sufficientemente leggera e veloce da essere eseguita su smartphone senza consumo di risorse eccessivo mantenendo comunque una precisione adeguata.

\section{Eye detection}
\label{sec:eyedetection}
La prima rete neurale riguarda il riconoscimento degli occhi di esseri umani, la quale risulterà essere il primo passo per eseguire anche un corretto riconoscimento delle pupille. Infatti, durante l'implementazione Android, l'output di questa rete, verrà usata come input per la seconda rete neurale incaricata del riconoscimento delle pupille. In questa fase ci limitiamo a creare una rete neurale in grado di riconoscere gli occhi.

\subsection{Dataset}
\label{sub:eyedataset}
Per il training del modello si è utilizzato un dataset pubblico rilasciato da Kaggle\footnote{\url{https://www.kaggle.com/datasets/pavelbiz/eyes-rtte}} contenente 5 mila immagini di sole pupille umane, comodo per il raggiungimento del nostro obiettivo di pupil detection.

Questa volta, per ottenere i file \textit{*.xml} di ogni immagine, abbiamo usufruito di un tool grafico chiamato \textbf{LabelImage}, scaricabile gratuitamente su Github\footnote{\url{https://github.com/tzutalin/labelImg}}. Questo tool ci permetteva di costruire i boxes attorno alla pupilla ottenendo in output per ogni immagine un file \textit{*.xml} contenente le informazioni necessarie: come il nome dell'oggetto su cui si è costruito attorno il box ("pupil"), e le relative coordinate geografiche dei 4 punti del box (xmin, ymin, xmax, ymax).

Prima dell'addestramento della rete era però necessario, usando questi nuovi dati \textit{*.xml}, generare i \textbf{TFRecord}, utili per TensorFlow.
Inoltre, per la corretta generazione dei \textbf{TFRecord}, si è reso necessario compilare il file \textit{labelmap.pbtxt} con al suo interno i nomi e gli id dei nostri oggetti da riconoscere: nel nostro caso un solo item di id pari a 1 e con nome "pupil".

\begin{figure}[htbp]
    \centering
    \includegraphics[scale=1]{ReteNeurale/PupilDetection/Dataset/Images/pupillabelmap_item.png}
    \caption{Pupil Detection labelmap.pbtxt}
    \label{fig:pupillabelmap}
\end{figure}


\subsection{Trainig}
\label{sub:eyetraining}
Si è quindi proceduto all’addestramento della rete tramite Python usando TensorFlow e le API di Keras. Si è reso necessario modificare il file \textit{pipeline.config} adattandolo alle nostre esigenze:

\begin{itemize}
    \item \textit{numclasses} che corrisponde al numero di item da identificare;
    \item \textit{path} vari da adattare al nostro workspace, tra cui il path per i checkpoint, per i record di input e per il labelmap.
\end{itemize}

Durante il procedimento di training si sono tenuti controllati i valori della nostra rete. In particolare, abbiamo usato \textbf{Tensorboard}: un framework grafico utile a capire l'andamento della nostra rete, infatti abbiamo avuto modo di controllare ad ogni training le performance della nostra rete di machine learning.

TensorBoard fornisce la visualizzazione e gli strumenti necessari per la sperimentazione del machine learning:

\begin{itemize}
    \item Monitoraggio e visualizzazione di metriche come perdita e precisione;
    \item Visualizzazione del grafico del modello (operazioni e livelli);
    \item Visualizzazione degli istogrammi di pesi, distorsioni o altri tensori man mano che cambiano nel tempo.
\end{itemize}

\begin{figure}[htbp]
    \centering
    \includegraphics[scale=0.7]{ReteNeurale/EyeDetection/Training/Images/learning_rate_both_edited.png}
    \caption{Tensorboard Learning\_rate}
    \label{fig:eyelearningrate}
\end{figure}

Si è ottenuto in output un modello addestrato e pronto all’uso.

\begin{figure}[htbp]
    \centering
    \subfigure[Test personaggio famoso]{
    \includegraphics[scale=0.16]{ReteNeurale/EyeDetection/Training/Images/einstein.png}
    \label{fig:testeyeinstein}
    }
    \hspace{5mm}
    \subfigure[Test multiplo]{
    \includegraphics[scale=0.16]{ReteNeurale/EyeDetection/Training/Images/gruppo.png}
    \label{fig:testeyegruppo}
    }
    \caption{Test Eye-Tracking}
    \label{fig:testeyetracking}
\end{figure}

Prima di testare direttamente su Android, abbiamo deciso prima di testare e analizzare la nostra nuova rete utilizzando la libreria Matplotlib, utile per visualizzare graficamente via shell i nostri risultati. In particolare abbiamo usato \textbf{Tkinter}, l'unico framework GUI incluso nella libreria standard di Python.

A fini di testing si è preso ad esempio il volto di un
personaggio pubblico, quello di Albert Einstein, e anche un'immagine contenente più persone in modo tale da poter verificare la correttezza e la precisione della rete anche in condizioni più difficili.

\subsection{TfLite}
\label{sub:eyetflite}
Dopo esserci accertati che la rete funzionasse correttamente e avesse dei livelli di precisione sopra una certa soglia, si è ottenuto in output un modello addestrato e pronto all’uso, che poi è stato convertito e quantizzato in un formato adatto ai sistemi embedded, quello di TensorFlow Lite.

\subsection{Metadata}
\label{sub:eyemetadata}
Per un corretto funzionamento della rete neurale all'interno di Android, TfLite richiede che venga generato anche un file Metadata che contiene le informazioni necessarie per pre-processare le immagini. Questo file si rende necessario in quanto i nostri tensori di input sono di tipo kTfLiteFloat32. È necessario creare un nuovo \textit{label\_map.txt} con al suo interno nella prima riga il nostro item per la detection: nel nostro caso \textit{"Human eye"}.

Una volta generato questo nuovo file metadata.tflite siamo pronti ad importarlo all'interno della nostra applicazione Android come file di Machine Learning "$ml$".

\section{Attività progettuale: Pupil detection}
\label{sec:gazedetection}
Per quanto riguarda invece la rete neurale incaricata al riconoscimento delle pupille, viene creata e addestrata in maniera simile alla nostra prima rete neurale. In questa fase ci limitiamo a crearla, in un secondo momento, durante l'implementazione Android, essa userà come input l'output della prima rete neurale dedicata agli occhi.

\subsection{Dataset}
\label{sub:gazedataset}
Per il training del modello si è utilizzato un dataset pubblico rilasciato da Kaggle\footnote{\url{https://www.kaggle.com/datasets/pavelbiz/eyes-rtte}} contenente 5 mila immagini di sole pupille umane, comodo per il raggiungimento del nostro obiettivo di pupil detection.

Questa volta, per ottenere i file \textit{*.xml} di ogni immagine, abbiamo usufruito di un tool grafico chiamato \textbf{LabelImage}, scaricabile gratuitamente su Github\footnote{\url{https://github.com/tzutalin/labelImg}}. Questo tool ci permetteva di costruire i boxes attorno alla pupilla ottenendo in output per ogni immagine un file \textit{*.xml} contenente le informazioni necessarie: come il nome dell'oggetto su cui si è costruito attorno il box ("pupil"), e le relative coordinate geografiche dei 4 punti del box (xmin, ymin, xmax, ymax).

Prima dell'addestramento della rete era però necessario, usando questi nuovi dati \textit{*.xml}, generare i \textbf{TFRecord}, utili per TensorFlow.
Inoltre, per la corretta generazione dei \textbf{TFRecord}, si è reso necessario compilare il file \textit{labelmap.pbtxt} con al suo interno i nomi e gli id dei nostri oggetti da riconoscere: nel nostro caso un solo item di id pari a 1 e con nome "pupil".

\begin{figure}[htbp]
    \centering
    \includegraphics[scale=1]{ReteNeurale/PupilDetection/Dataset/Images/pupillabelmap_item.png}
    \caption{Pupil Detection labelmap.pbtxt}
    \label{fig:pupillabelmap}
\end{figure}


\subsection{Trainig}
\label{sub:gazetraining}
Si è quindi proceduto all’addestramento della rete tramite Python usando TensorFlow e le API di Keras. Si è reso necessario modificare il file \textit{pipeline.config} adattandolo alle nostre esigenze:

\begin{itemize}
    \item \textit{numclasses} che corrisponde al numero di item da identificare;
    \item \textit{path} vari da adattare al nostro workspace, tra cui il path per i checkpoint, per i record di input e per il labelmap.
\end{itemize}

Durante il procedimento di training si sono tenuti controllati i valori della nostra rete. In particolare, abbiamo usato \textbf{Tensorboard}: un framework grafico utile a capire l'andamento della nostra rete, infatti abbiamo avuto modo di controllare ad ogni training le performance della nostra rete di machine learning.

TensorBoard fornisce la visualizzazione e gli strumenti necessari per la sperimentazione del machine learning:

\begin{itemize}
    \item Monitoraggio e visualizzazione di metriche come perdita e precisione;
    \item Visualizzazione del grafico del modello (operazioni e livelli);
    \item Visualizzazione degli istogrammi di pesi, distorsioni o altri tensori man mano che cambiano nel tempo.
\end{itemize}

\begin{figure}[htbp]
    \centering
    \includegraphics[scale=0.7]{ReteNeurale/EyeDetection/Training/Images/learning_rate_both_edited.png}
    \caption{Tensorboard Learning\_rate}
    \label{fig:eyelearningrate}
\end{figure}

Si è ottenuto in output un modello addestrato e pronto all’uso.

\begin{figure}[htbp]
    \centering
    \subfigure[Test personaggio famoso]{
    \includegraphics[scale=0.16]{ReteNeurale/EyeDetection/Training/Images/einstein.png}
    \label{fig:testeyeinstein}
    }
    \hspace{5mm}
    \subfigure[Test multiplo]{
    \includegraphics[scale=0.16]{ReteNeurale/EyeDetection/Training/Images/gruppo.png}
    \label{fig:testeyegruppo}
    }
    \caption{Test Eye-Tracking}
    \label{fig:testeyetracking}
\end{figure}

Prima di testare direttamente su Android, abbiamo deciso prima di testare e analizzare la nostra nuova rete utilizzando la libreria Matplotlib, utile per visualizzare graficamente via shell i nostri risultati. In particolare abbiamo usato \textbf{Tkinter}, l'unico framework GUI incluso nella libreria standard di Python.

A fini di testing si è preso ad esempio il volto di un
personaggio pubblico, quello di Albert Einstein, e anche un'immagine contenente più persone in modo tale da poter verificare la correttezza e la precisione della rete anche in condizioni più difficili.

\subsection{TfLite}
\label{sub:gazetflite}
Dopo esserci accertati che la rete funzionasse correttamente e avesse dei livelli di precisione sopra una certa soglia, si è ottenuto in output un modello addestrato e pronto all’uso, che poi è stato convertito e quantizzato in un formato adatto ai sistemi embedded, quello di TensorFlow Lite.

\subsection{Metadata}
\label{sub:gazemetadata}
Per un corretto funzionamento della rete neurale all'interno di Android, TfLite richiede che venga generato anche un file Metadata che contiene le informazioni necessarie per pre-processare le immagini. Questo file si rende necessario in quanto i nostri tensori di input sono di tipo kTfLiteFloat32. È necessario creare un nuovo \textit{label\_map.txt} con al suo interno nella prima riga il nostro item per la detection: nel nostro caso \textit{"Human eye"}.

Una volta generato questo nuovo file metadata.tflite siamo pronti ad importarlo all'interno della nostra applicazione Android come file di Machine Learning "$ml$".