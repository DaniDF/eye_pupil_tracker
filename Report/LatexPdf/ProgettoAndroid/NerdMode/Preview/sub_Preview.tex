Il bottone "\textit{Start preview}" (vedi figura \ref{fig:nerdmodelayout}) avvia l'acquisizione dalla fotocamera del dispositivo, come fotocamera predefinita viene utilizzata quella "interna" ovvero quella rivolta verso l'utente. È possibile anche usare la fotocamera "esterna" mediante il bottone "\textit{Switch Camera}" (vedi \ref{sub:switchcamera}).

La dimensione della \textit{preview view} deve essere adattata alla dimensione del sensore della camera e quindi dell'immagine risultante. Queste dimensioni dipendono sia dal dispositivo su cui viene eseguita l'applicazione sia da quale fotocamera viene selezionata, pertanto dinamicamente l'applicazione deve adattare le dimensioni della \textit{preview view} e coerentemente il layer su cui vengono disegnati i rettangoli di tracking degli occhi. Se questa operazione non venisse effettuata o non eseguita dinamicamente la rete opererebbe comunque in maniera corretta ma la visualizzazione dei risultati risulterebbe completamente errata mostrando l'occhio in in punto dello schermo e il corrispondente rettangolo in una posizione diversa.